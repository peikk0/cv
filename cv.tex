%!TEX TS-program = xelatex

\documentclass[]{friggeri-cv}

\begin{document}

\hypersetup{
  pdfauthor={Pierre Guinoiseau},
  pdftitle={Pierre Guinoiseau --- Site Reliability Engineer}
}

\header{Pierre Guinoiseau}
       {Site Reliability Engineer}



% In the aside, each new line forces a line break
\begin{aside}
  \section{contact}
    \href{mailto:pierre@guinoiseau.nz}{pierre@guinoiseau.nz}
    +64 29 126 7846
    \href{https://linkedin.com/in/pierreguinoiseau}{linkedin://pierreguinoiseau}
  \section{languages}
    \textit{French:} native
    \textit{English:} fluent
    \textit{Spanish:} notions
    \textit{Te Reo Māori:} learning
  \section{programming}
    Go, Python, {\color{red} $\varheartsuit$} Ruby
\end{aside}



\section{summary}

I'm a 32 years old Frenchman in New Zealand with 12+ years experience as a
Linux/Unix systems engineer, Site Reliability Engineer / DevOps engineer and
Go, Python, Ruby developer. I'm looking for my next adventure!



\section{experience}

\begin{entrylist}

  \entry
    {2018--2020}
    {Magic Leap}
    {Wellington, New Zealand}
    {
      \textit{Lead Site Reliability Engineer} \\
      \\
      Magic Leap, Inc. is an American startup company that released a
      head-mounted virtual retinal display, called Magic Leap One, which
      superimposes 3D computer-generated imagery over real world objects, by
      ``projecting a digital light field into the user's eye'', involving
      technologies potentially suited to applications in augmented reality and
      computer vision. \\
      \\
      As a Lead Site Reliability Engineer, I worked remotely in an
      international team to ensure the safe, swift, and reliable delivery of
      services to the company's customers. My role combined software and
      systems engineering to deliver highly scalable, distributed, fault
      tolerant systems.
      \\
      My responsibilities included:
      \begin{itemize}
        \item developing solutions to increase service stability through
              automation and process re-engineering;
        \item building and supporting tools and systems that software engineers
              use to deploy their software into production,
              contributing to a great extent to an internal build tool written
              in Go wrapping Terraform operations;
        \item participating in rotating on-call duties in a global
              24$\times$7$\times$365 team;
        \item updating job knowledge by studying state-of-the-art tools and
              techniques;
        \item helping development teams operationalize their efforts to enable
              self-ownership of production services.
      \end{itemize}
    }

  \entry
    {2015--2018}
    {Catalyst IT}
    {Wellington, New Zealand}
    {
      \textit{DevOps engineer / PHP and Python developer} \\
      \\
      Catalyst IT is a global team of skilled open source technologists,
      specializing in developing, designing and supporting enterprise grade
      systems using open source technologies. \\
      \\
      As a DevOps engineer, I worked in the eLearning team, delivering,
      customizing and hosting Moodle and Totara LMS sites for schools,
      universities, corporations and government organisations. My role
      consisted in building and maintaining our CI and deployment tools and
      helping with infrastructure changes, as well as doing some backend
      development on those sites. I also helped with some web development on
      PaCT (Progress and Consistency Tool), built with Django.
    }

  \entry
    {2010--2015}
    {HR Team / Subcontractor for M6 Web}
    {Lyon, France}
    {
      \textit{Linux/Unix systems engineer / DevOps engineer} \\
      \\
      M6 Web, the M6 Group’s subsidiary responsible for developing new
      technologies, is structured around several operational platforms:
      websites of TV channels, offering Catch-up TV and VOD, thematic web
      portals, comparison shopping websites, mobile phones with M6’s Mobile by
      Orange offer, and finally games and channel interactivity. \\
      \\
      As a member of a small Linux/Unix systems engineering team working
      together with a remote ops team, I was in charge of the hosting
      infrastructure, spread across 2 datacenters and AWS, and the local
      development platform, for a total of 400+ physical and virtual
      heterogeneous hosts hosting 150+ websites and webservices. \\
      \\
      As we started to adopt agile methods, I focused mainly on providing
      developers an environment and tools that would enable them to develop and
      deploy their products in an automated, fast and reliable way.
    }

  \entry
    {2008--2010}
    {Pilot Systems}
    {Lyon \& Paris, France (remote work)}
    {
        \textit{Python/Zope/Plone developer and Linux/Unix system
                administrator} \\
        \\
        Pilot Systems is an Open Source Integrator, specializing in Python
        solutions, like Zope, Plone, Django, etc. \\
        \\
        As a developer, I worked on the development and maintenance of intranet
        and extranet solutions for several customers, based on Zope (2.x) and
        Plone (2.x and 3.x), as well as the maintenance and a major migration
        of a major French newspaper's website. \\
        \\
        As a system administrator, I maintained and modernised the company's
        hosting infrastructure, and provided systems engineering consulting for
        some customers.
    }

  \entry
    {2007--2008}
    {Eliot}
    {Voiron, France}
    {
        \textit{Web developer and embedded system developer --- Part-time
        internship} \\
        \\
        Eliot designs and produces modular on-board telematic Solutions for
        transport (goods and passenger), logistics and industrial companies. \\
        \\
        As a student intern, I had 2 different missions in parallel:
        \begin{itemize}
          \item developed an API around a proprietary mapping, routing and
                geocoding server (deCarta DDS) that could be used in a
                backoffice software;
          \item tested a next-generation on-board terminal running Embedded
                Linux OS on an ARM platform.
        \end{itemize}
    }

\end{entrylist}



\section{education}

\begin{entrylist}

\entry
  {2007--2008}
  {Bachelor's Degree {\normalfont in Computer Science}}
  {Université Pierre Mendès-France (Grenoble II)}
  {
    Licence professionnelle ``Systèmes Informatiques et Logiciels''
    (Information Systems and Softwares)
  }

\entry
  {2005--2007}
  {Associate's Degree {\normalfont in Computer Science}}
  {Université Pierre Mendès-France (Grenoble II)}
  {
    DUT Informatique (Higher National Diploma in Technology / Computer
    Science)
  }

\end{entrylist}



\newpage

\section{skills}

\begin{itemize}
  \item \textbf{Programming languages:}
        Go, Python, {\color{red} $\varheartsuit$} Ruby
  \item \textbf{Operating systems:}
        Linux (Gentoo, Debian, Ubuntu, RedHat), {\color{red} $\varheartsuit$} FreeBSD, OpenBSD
  \item \textbf{Cloud computing platforms:}
        AWS, GCP
  \item \textbf{Virtualization:}
        VMWare ESX(i), Xen
  \item \textbf{Infrastructure as Code:}
        CloudFormation, Terraform
  \item \textbf{Containerization:}
        Docker
  \item \textbf{Configuration management:}
        Puppet + MCollective, Chef, Ansible
  \item \textbf{CI:}
        {\color{red} $\varheartsuit$} Gitlab CI, Bitbucket Pipelines, Concourse, Jenkins, GoCD
  \item \textbf{SCM platforms:}
        {\color{red} $\varheartsuit$} GitLab, GitHub, Bitbucket
  \item \textbf{Logging, monitoring:}
        Datadog, Sentry, SumoLogic, Prometheus, StatsD + Graphite + Grafana,
        Cacti, Nagios/Icinga
\end{itemize}



\section{interests}

Playing guitar, drumming, photography, longboarding, hiking, martial arts,
sharing craft beers with my friends.



\section{referees}

Available on request.

\end{document}
